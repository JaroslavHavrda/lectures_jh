\documentclass{beamer}

\mode<presentation>

\usecolortheme{magpie}

\title{Jak origami vyřešilo problém Řeckých bohů}
\author{Jaroslav Havrda}
\date[AKICon]{AKICON 2025}

\begin{document}

\begin{frame}
	\titlepage
\end{frame}

\begin{frame}
	\frametitle{Problém z Delf}
	\begin{itemize}
		\item Občané Délosu vyhledali orákulum v Delfách s otázkou, jak zastavit epidemii, kterou na jejich město uvalil Apolón.
		\pause
		\item Orákulum odpovědělo, že musí zdvojnásobit velikost oltáýře a tím si Apolóna usmířit. Oltář při tom musel zachovat tvar krychle.
		\pause

		\item Zlé jazyky v podobě Plútarcha tvrdí, že ve skutečnosti hledali řešení svých politických rozbrojů.
		\pause
		\item Úkol zdvojení krychle je pak měl odpoutat od jejich hádek a přivést k studiu matematiky.
		\pause
	\end{itemize}
\end{frame}

\begin{frame}
	\frametitle{Povolené konstrukce}
	\begin{enumerate}
		\item Nakreslit přímku mezi dvěma body
		\item Narýsovat kružnici danou středem a jedním bodem.
		\item Označit průsečík dvou přímek
		\item Označit průsečík přímky a kružnice
		\item Označit průsečíky dvou kružnic.
		\item Udělat kolmici
		\item Udělat rovnoběžky
	\end{enumerate}
\end{frame}

\begin{frame}
	\frametitle{Jaké délky umím udělat?}
	\begin{itemize}
		\item Celá čísla
		\item Zlomky
		\item druhá odmocnina
	\end{itemize}
\end{frame}

\begin{frame}
	\frametitle{Proč je to těžké?}
	Přímka je reprezentována lineární rovnicí.
	Kružnice je reprezentována kvadratickou rovnicí.
	Potřebujeme vyřešit rovnici \(x^3 = 2\).
	Protože umíme sestrojit pouze kvadratickou rovnici, musíme převést \(x^3=2\) na kvadratickou rovnici.
\end{frame}

\end{document}